\documentclass[a4paper]{article}
\usepackage[utf8]{inputenc}
\usepackage[T1]{fontenc}
\usepackage[spanish]{babel}
\usepackage{listings}
\usepackage{hyperref}
\usepackage{graphicx}
\usepackage{gensymb}
\usepackage{comment}
\usepackage{listings}
\usepackage[official]{eurosym}
\usepackage{color}
\definecolor{lightgray}{rgb}{.9,.9,.9}
\definecolor{darkgray}{rgb}{.4,.4,.4}
\definecolor{purple}{rgb}{0.65, 0.12, 0.82}

\lstdefinelanguage{JavaScript}{
  keywords={typeof, new, true, false, catch, function, return, null, catch, switch, var, if, in, while, do, else, case, break},
  keywordstyle=\color{blue}\bfseries,
  ndkeywords={class, export, boolean, throw, implements, import, this},
  ndkeywordstyle=\color{darkgray}\bfseries,
  identifierstyle=\color{black},
  sensitive=false,
  comment=[l]{//},
  morecomment=[s]{/*}{*/},
  commentstyle=\color{purple}\ttfamily,
  stringstyle=\color{red}\ttfamily,
  morestring=[b]',
  morestring=[b]"
}

\lstset{
   language=JavaScript,
   backgroundcolor=\color{lightgray},
   extendedchars=true,
   basicstyle=\footnotesize\ttfamily,
   showstringspaces=false,
   showspaces=false,
   numbers=left,
   numberstyle=\footnotesize,
   numbersep=9pt,
   tabsize=2,
   breaklines=true,
   showtabs=false,
   captionpos=b
}



\title{PLD: Propuesta del Juego}
\author{Jose Miguel Colella (Y1453965B) y\\José Manuel Gómez González (45920481-S)}


\begin{document}

\maketitle

\section{Propuesta}

\begin{description}
  \item[Título provisional del videojuego] \hfill \\
  Nuestro juego tiene el siguiente título provisional: \textbf{Treasure Hunter}
  \item[Descripción general] \hfill \\
   Se tratará de un juego en 2D de estrategia por turnos. El
escenario es el interior de un complejo secreto (habitaciones, pasillos,
corredores ...). El objetivo consiste en robar un objeto del que se
conocen posibles ubicaciones que el jugador deberá probar para
finalmente huir con el objeto. Un jugador adversario controlado por IA
tendrá el mismo objetivo. Adicionalmente, otra IA controlará varios
agentes encargados de vigilar y atacar a cualquier intruso (el propio
jugador y la otra IA). Como elementos adicionales sobre el escenario
tenemos: interruptores de control, puertas y lasers. Los interruptores
de control abren/cierran (o activan/desactivan) puertas y lasers. Los
lasers a la altura de la cintura solo afectan a los jugadores (usuario e
IA) quitandoles puntos de vida. Las puertas afectan a todos al bloquear
el paso. La dificultad vendrá dada por la IA y por el propio mapa.
Otro punto interesante aunque complementario sería la
posibilidad de generar mapas de manera automática.
  \item[Género] \hfill \\
  \href{http://en.wikipedia.org/wiki/Video_game_genre#Turn-based_strategy}{Turn Based Strategy Game}
  \item[Audiencia]
  Cualquier tipo de persona es capaz de juegar el juego aunque tiene una tematica de puede
  ser influencial en jovenes

  \item[3 videojuegos del mismo segmeto] \hfill \\
    \begin{enumerate}
      \item
      \begin{description}
        \item[Título] \hfill \\
        Rome Total War 2
        \item[Compañia] \hfill \\
        Creative Assembly \\Sega
        \item[Plataforma] \hfill \\
        PC (Microsoft Windows, Linux)
        \item[Comercialización] \hfill \\
        DVD, Descarga por Steam 54,99 \euro{}
        \item[Página web] \hfill \\
        \hyperlink{http://www.totalwar.com/en_us/rome2/}{Rome Total War 2}
        \item[Capturas de pantalla] \hfill \\
        \begin{figure}[ht!]
        \centering
        \includegraphics[width=90mm]{./Total-War-Rome-2-11.jpg}
        \caption{Campaña Turn Based Strategy}
        \label{overflow}
        \end{figure}
        \begin{figure}[ht!]
        \centering
        \includegraphics[width=90mm]{./images.jpeg}
        \caption{El juegador controla ciudades y pelea por establecer su dominio}
        \label{overflow2}
        \end{figure}
        \item[3 características destacadas] \hfill \\
          \begin{enumerate}
            \item Las gráficas del juego son extraordinarias
            \item La inteligencia artificial es robusta
            \item Basado en hechos reales
          \end{enumerate}
        \item[3 limitaciones observadas] \hfill \\
          \begin{enumerate}
            \item Cuando se juega en modo difícil la inteligencia artificial toma mucho tiempo
            en hacer una decisión
            \item Unas veces IA no toma decisiones que beneficia el grupo que representa
            \item Tienes que tener capacidades gráficas muy buenas para tener muy buenas FPS (Frames per seconds).
          \end{enumerate}
      \end{description}
      \item
      \begin{description}
        \item[Título] \hfill \\
        Sid Meier's Civilization V
        \item[Compañia] \hfill \\
        Firaxis Game\\ 2K Games \& Aspyr
        \item[Plataforma] \hfill \\
        PC (Microsoft Windows), OSX, \href{http://en.wikipedia.org/wiki/OnLive}{OnLive}
        \item[Comercialización] \hfill \\
        DVD, Descarga en Steam 29.99 \euro{}, Cloud Computing
        \item[Página web] \hfill \\
        \hyperlink{http://www.civilization5.com/}{Rome Total War 2}
        \item[Capturas de pantalla] \hfill \\
        \begin{figure}[ht!]
        \centering
        \includegraphics[width=90mm]{./civilization.jpg}
        \caption{Como se puede ver, hay un mapa donde el usuario y los enemigos hacen ciertos movimientos en sus turnos}
        \label{overflow3}
        \end{figure}
        \item[3 características destacadas] \hfill \\
          \begin{enumerate}
            \item Las condiciones de victoria para que el jugador tiene que ver con investigación, exploración, diplomacia, expansión, desarrollo económico, etc\ldots.
            \item Tiene que ver con eventos de la historia
            \item Tiene capacidades de multijugador
          \end{enumerate}
        \item[3 limitaciones observadas] \hfill \\
          \begin{enumerate}
            \item \href{http://www.ign.com/articles/2010/09/20/civilization-v-review?page=2}{La IA es muy agresiva forzando al jugador a ganar por fuerza bruta.}
            \item Gráficos no muy buenos
            \item Cuando llegas a un punto que dominas la partida, la IA juega un juego conservativo
          \end{enumerate}
      \end{description}
      \begin{description}
        \item[Título] \hfill \\
        The Battle for Wesnoth
        \item[Compañia] \hfill \\
        -
        \item[Plataforma] \hfill \\
        PC (Linux, Windows, MacOSX)
        \item[Comercialización] \hfill \\
        Digital a 0 \euro{} la descarga
        \item[Página web] \hfill \\
        http://www.wesnoth.org/
        \item[Capturas de pantalla] \hfill \\
        \begin{figure}[ht!]
        \centering
        \includegraphics[width=90mm]{./wesnoth1.jpg}
        \caption{Juego con su motor gráfico}
        \label{overflow4}
        \end{figure}
        \item[3 características destacadas] \hfill \\
          \begin{enumerate}
            \item Entorno de desarrollo y lenguaje de marcado para creación de contenido.
            \item Gran cantidad de contenido desarrollado por la comunidad de manera activa.
            \item Posibilidad de alterar el comportamiento de la AI
          \end{enumerate}
        \item[3 limitaciones observadas] \hfill \\
          \begin{enumerate}
            \item Narración en modo campaña basada en texto.
            \item Requiere pasar por un tutorial.
            \item La posición de las unidades está muy discretizada.
          \end{enumerate}
      \end{description}

    \end{enumerate}

\end{description}

\end{document}

