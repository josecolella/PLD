\documentclass[a4paper]{article}
\usepackage[utf8]{inputenc}
\usepackage[T1]{fontenc}
\usepackage[spanish]{babel}
\usepackage{listings}
\usepackage{hyperref} 
\usepackage{gensymb}
\usepackage{comment}
\usepackage{listings}
\usepackage{color}
\definecolor{lightgray}{rgb}{.9,.9,.9}
\definecolor{darkgray}{rgb}{.4,.4,.4}
\definecolor{purple}{rgb}{0.65, 0.12, 0.82}

\lstdefinelanguage{JavaScript}{
  keywords={typeof, new, true, false, catch, function, return, null, catch, switch, var, if, in, while, do, else, case, break},
  keywordstyle=\color{blue}\bfseries,
  ndkeywords={class, export, boolean, throw, implements, import, this},
  ndkeywordstyle=\color{darkgray}\bfseries,
  identifierstyle=\color{black},
  sensitive=false,
  comment=[l]{//},
  morecomment=[s]{/*}{*/},
  commentstyle=\color{purple}\ttfamily,
  stringstyle=\color{red}\ttfamily,
  morestring=[b]',
  morestring=[b]"
}

\lstset{
   language=JavaScript,
   backgroundcolor=\color{lightgray},
   extendedchars=true,
   basicstyle=\footnotesize\ttfamily,
   showstringspaces=false,
   showspaces=false,
   numbers=left,
   numberstyle=\footnotesize,
   numbersep=9pt,
   tabsize=2,
   breaklines=true,
   showtabs=false,
   captionpos=b
}



\title{PLD: Idea del Juego}
\author{Jose Miguel Colella y\\José Manuel Gómez González}

\begin{document}

\maketitle

Un juego en dos dimensiones por turno. Dos jugados básicos con un tercer 

Un juego de infiltración donde un jugador tiene que entrar en un sitio, y la IA te lo impide.
El jugador se encuentra dentro de un escenario en el cual tiene que encontrar un objeto especifico y salir de la escena. El jugador tiene que encontrar este objetivo en un tiempo determinado. Para enforzar este tiempo, hay una plaga que se va comiendo el escenario, hasta que el tiempo se acaba.
También hay otro enemigo, que tiene un grado de visión, cual objetivo es destruir el jugador principal. Este enemigo también tiene la habilidad de coger el objeto en cuestion para el mismo.

La plaga se vaya moviendo cada vez que el usuario termina su turno. 

\section{Requisitos}


\section{Plataforma}


\section{Herramientas de Desarrollo}



\section{Integración de Inteligencia Artificial}




\end{document}

