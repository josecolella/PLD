\documentclass[a4paper]{article}
\usepackage[utf8]{inputenc}
\usepackage[T1]{fontenc}
\usepackage[spanish]{babel}
\usepackage{listings}
\usepackage{hyperref}
\usepackage{graphicx}
\usepackage{gensymb}
\usepackage{comment}
\usepackage{listings}
\usepackage[official]{eurosym}
\usepackage{color}
\definecolor{lightgray}{rgb}{.9,.9,.9}
\definecolor{darkgray}{rgb}{.4,.4,.4}
\definecolor{purple}{rgb}{0.65, 0.12, 0.82}

\lstdefinelanguage{JavaScript}{
  keywords={typeof, new, true, false, catch, function, return, null, catch, switch, var, if, in, while, do, else, case, break},
  keywordstyle=\color{blue}\bfseries,
  ndkeywords={class, export, boolean, throw, implements, import, this},
  ndkeywordstyle=\color{darkgray}\bfseries,
  identifierstyle=\color{black},
  sensitive=false,
  comment=[l]{//},
  morecomment=[s]{/*}{*/},
  commentstyle=\color{purple}\ttfamily,
  stringstyle=\color{red}\ttfamily,
  morestring=[b]',
  morestring=[b]"
}

\lstset{
   language=JavaScript,
   backgroundcolor=\color{lightgray},
   extendedchars=true,
   basicstyle=\footnotesize\ttfamily,
   showstringspaces=false,
   showspaces=false,
   numbers=left,
   numberstyle=\footnotesize,
   numbersep=9pt,
   tabsize=2,
   breaklines=true,
   showtabs=false,
   captionpos=b
}



\title{PLD: Herramienta}
\author{Jose Miguel Colella (Y1453965B) y\\José Manuel Gómez González (45920481-S)}


\begin{document}

\maketitle

\section{Alternativas}
\begin{description}
    \item[PyGame]
    \begin{itemize}
        \item Página: http://pygame.org
        \item Categoría: biblioteca
        \item Coste: Licencia GPL
        \item 3 características destacadas que pueden resultar útiles
            \begin{enumerate}
                \item Multiplataforma
                \item Se basa en un lenguaje de prototipado (Python)
                \item Facil integración de IA
            \end{enumerate}
        \item 3 limitaciones observadas de cara al videojuego propuesto
            \begin{enumerate}
                \item Se tiene que programar, no hay ninguna programación visual
                \item Se tiene que integrar un modulo externo para integración de red
                \item Motor de gráficos basado en sprite
            \end{enumerate}
    \end{itemize}
    \item[Moleculejs]
    \begin{itemize}
        \item Página: http://moleculejs.net
        \item Categoría: libreria
        \item Coste: Licencia AGPLv3
        \item 3 características destacadas que pueden resultar útiles
            \begin{enumerate}
                \item Esta integrado en Web asi que todas las plataformas que soportan HTML5 pueden usarla
                \item Soporte para touch
                \item Una biblioteca ligera y no usa librerias externas
            \end{enumerate}
        \item 3 limitaciones observadas de cara al videojuego propuesto
            \begin{enumerate}
                \item Un proyecto nuevo donde no hay una grande comunidad
                \item Documentación mala
                \item No incluye nada sobre multijugador o IA
            \end{enumerate}
    \end{itemize}
    \item[Three.js]
    \begin{itemize}
        \item Página: http://threejs.org
        \item Categoría: libreria
        \item Coste: Licencia MIT
        \item 3 características destacadas que pueden resultar útiles
            \begin{enumerate}
                \item Documentación con varios ejemplos
                \item Herramienta gráfica de apoyo
                \item Tiene soporte para 2D y 3D
            \end{enumerate}
        \item 3 limitaciones observadas de cara al videojuego propuesto
            \begin{enumerate}
                \item Se tiene que usar una libreria externa para integrar sonido
                \item Se tiene que integrar modulos externos para IA
                \item Se tiene que instalar modulo externo para integración de multiplayer.
            \end{enumerate}
    \end{itemize}




\end{description}

\section{Decisión final}
\subsection{Tabla resumen}
A continuación se enumeran todas las características tenidas en cuenta hasta ahora con el fin de confeccionar la tabla:
\begin{enumerate}
\item Multiplataforma
\item Orientado a la Web
\item Otros dispositivos de interacción aparte de teclado y ratón
\item Integración IA
\item Integración Red
\item Integración Sonido
\item Soporte 2D
\item Soporte 3D
\item Prototipado rápido
\item Herramientas visuales de apoyo
\item Documentación
\end{enumerate}

La tabla queda de la siguiente forma:
\begin{center}
\begin{tabular}{ |l| *{11}{c|} }
\hline
Alternativa  & 1 & 2 & 3 & 4 & 5 & 6 & 7 & 8 & 9 & 10 & 11 \\ \hline \hline
PyGame & +& ---& +& ---& ---& ++& ++& ---& ++& ---& ++\\
Moleculejs & ++& ++& ---& ---& ---& ---& ---& ---& ++& ++& ---\\
Three.js & ++& ++& ---& ---& ---& ---& ++& ++& ++& +& ++\\
\hline
\end{tabular}
\end{center}

\subsection{Conclusión}
A la vista de los datos de la tabla, la mejor alternativa a considerar para el desarrollo del juego es \emph{Three.js}.

\end{document}

