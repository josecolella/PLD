\documentclass{article}
\usepackage[utf8]{inputenc}
\usepackage{gensymb}
\usepackage{tikz-uml}

\title{Programación Lúdica: Treasure Hunters}
\author{José Miguel Colella Carbonara\\José Manuel Gómez}




\begin{document}

\maketitle

\section{Introducción}

En este documento queremos denotar los aspectos más relevantes en el desarrollo
de nuestro juego \emph{Treasure Hunters} usando como base la herramienta \textit{pygame}
que proporciona los mecanismos basicos para construir un videojuego usando \textbf{python}. \hfill \\

Las secciones que vamos a describir las opiniones sobre \textit{pygame} como herramienta
para comenzar un videojuego, el diseño del juego en base a como se ha estructurado, los mecanismo
de inteligencia artificial, y nuestras conclusiones sobre el camino que hemos recorrido desde el diseño
hasta la finalidad de tener un juego funcional.
k

\section{Pygame}



\section{Diseño del Juego}



Nuestro diseño para este juego ha utilizado el paradigma de programación dirigida
a objectos. Hemos usado las ventajas de la herencia para crear objetos escalables, faciles
de usar y extender.

Para los personajes que forman parte del juego que incluye el \textit{MainCharacter},
\textit{Enemy}, y \textit{Robot}, todos se extienden de una clase llamada \textit{Character}
que proporciona los mecanismos que tiene que tener todo objecto que se mueve e interactua
con el jugador principal. Dentro de dicha clase se han definidos todas las acciones de movimiento,
de como se visualiza dicho movimiento al usuario (cambio de imagenes...), mecanismos de disparo,
condiciones de victoria, etc\ldots


\section{Inteligencia Artificial}



\section{Conclusión}




\end{document}
