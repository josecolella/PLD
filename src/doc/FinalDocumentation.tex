\documentclass{article}
\usepackage[utf8]{inputenc}
\usepackage{gensymb}

\title{Programación Lúdica: Treasure Hunters}
\author{José Miguel Colella Carbonara\\José Manuel Gómez}




\begin{document}

\maketitle

\section{Introducción}

En este documento queremos denotar los aspectos más relevantes en el desarrollo
de nuestro juego \emph{Treasure Hunters} usando como base la herramienta \textit{pygame}
que proporciona los mecanismos basicos para construir un videojuego usando \textbf{python}. \hfill \\

Las secciones que vamos a destacar son las opiniones sobre \textit{pygame} como herramienta
para el desarrollo de un videojuego, el diseño del juego en base a como se ha estructurado, los mecanismo
de inteligencia artificial, cosas particulares del juego, y nuestras conclusiones sobre el
camino que hemos recorrido desde el diseño hasta la finalidad de tener un juego funcional.


\section{Pygame}

\textit{Pygame} proporciona un fundación sólida para el diseño de un videojuego
y los objectos que estarán presentes. En base a colisiones, creación de menus,
gestion de imagenes, y sprites pygame facilita la construcción de objetos
para la creación de un videojuego.
Una cosa de notar es que por ejemplo, mecanismos de guardar, cargar, estructuración
de videojuego, y la inteligencia artificial la tiene que crear el desarrollador
desde nada.

\section{Diseño del Juego}

\indent Nuestro diseño para este juego ha utilizado el paradigma de programación dirigida
a objectos. Hemos usado las ventajas de la herencia para crear objetos escalables, faciles
de usar y extender.

Para los personajes que forman parte del juego que incluye el \textit{MainCharacter},
\textit{Enemy}, y \textit{Robot}, todos se extienden de una clase llamada \textit{Character}
que proporciona los mecanismos de movimiento, disparo, interacción con otros objetos en el juego, etc\ldots \hfill \\


\section{Inteligencia Artificial}

Para la construcción de la Inteligencía Artificial se ha pensado en una construcción de IA multinivel,
en el cual tenemos un nivel que tiene acciones de alto nivel:
    \begin{itemize}
        \item change\_zone(asset, new zone)
        \item change\_gun(asset)
        \item pick\_up(asset, object)
        \item drop(asset)
        \item hurt(asset, asset)
        \item switch(asset, lever)
    \end{itemize}

mientras que el otro nivel se encarga de las acciones de bajo nivel:
    \begin{itemize}
        \item fire(direction)
        \item move(direction)
        \item \ldots
    \end{itemize}

Para hacer esto se ha separado las responsabilidades en tres clases:
    \begin{enumerate}
        \item Agent
        \item AgentServer
        \item Think
    \end{enumerate}

La idea es tener a AgentServer construye objetos Think y Agent interconectados
y mantiene una copia del estado del juego lista para ser procesada por los mecanismos de IA.
Se ha usado el paradigma de programación concurrente para poder separar los procesos de la IA de la hebra principal.

\section{Notas Adicionales}

Unas de las cosas que quisieramos destacar es el sistema de guardar y cargar y el
potencial que ofrece para una futura integración multijugador red. Se ha usado
el sistema de intercambio ligero, \textit{JSON}, en el cual se guarda el estado actual del
juego, almacenando las cosas más importantes; coordenadas, información sobre el
objeto a coger, sobre los enemigos, etc\ldots \hfill \\
Para un sistema multijugador se transferiría dicha información en formato \textit{JSON},
a los otros jugadores



\section{Conclusión}

\subsection{José Colella}

Por mi parte, pienso que para haber tenido que construir la estructura del juego,
diseñando un modelo escalable, y creando la Inteligencia Artificial desde nada,
hemos creado un juego con gran potencial, y que en su estado actual ofrece un opción
completa. \hfill \\
Desde tener un menu, uno para acceder al juego ó cargar un estado previo y otro
que se usa para acceder desde el juego y guardar el estado, crear música para menu y
videojuego, se ha creado un producto lo más completo posible. \\
Estoy muy satisfecho con el producto que hemos sacado a luz con mi compañero

\subsection{José Manuel Gomez}



\end{document}
